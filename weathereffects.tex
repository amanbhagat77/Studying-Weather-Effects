\PassOptionsToPackage{unicode=true}{hyperref} % options for packages loaded elsewhere
\PassOptionsToPackage{hyphens}{url}
%
\documentclass[
]{article}
\usepackage{lmodern}
\usepackage{amssymb,amsmath}
\usepackage{ifxetex,ifluatex}
\ifnum 0\ifxetex 1\fi\ifluatex 1\fi=0 % if pdftex
  \usepackage[T1]{fontenc}
  \usepackage[utf8]{inputenc}
  \usepackage{textcomp} % provides euro and other symbols
\else % if luatex or xelatex
  \usepackage{unicode-math}
  \defaultfontfeatures{Scale=MatchLowercase}
  \defaultfontfeatures[\rmfamily]{Ligatures=TeX,Scale=1}
\fi
% use upquote if available, for straight quotes in verbatim environments
\IfFileExists{upquote.sty}{\usepackage{upquote}}{}
\IfFileExists{microtype.sty}{% use microtype if available
  \usepackage[]{microtype}
  \UseMicrotypeSet[protrusion]{basicmath} % disable protrusion for tt fonts
}{}
\makeatletter
\@ifundefined{KOMAClassName}{% if non-KOMA class
  \IfFileExists{parskip.sty}{%
    \usepackage{parskip}
  }{% else
    \setlength{\parindent}{0pt}
    \setlength{\parskip}{6pt plus 2pt minus 1pt}}
}{% if KOMA class
  \KOMAoptions{parskip=half}}
\makeatother
\usepackage{xcolor}
\IfFileExists{xurl.sty}{\usepackage{xurl}}{} % add URL line breaks if available
\IfFileExists{bookmark.sty}{\usepackage{bookmark}}{\usepackage{hyperref}}
\hypersetup{
  pdftitle={Studying Various Effects of Severe Weather Events},
  pdfauthor={Aman Bhagat},
  pdfborder={0 0 0},
  breaklinks=true}
\urlstyle{same}  % don't use monospace font for urls
\usepackage[margin=1in]{geometry}
\usepackage{color}
\usepackage{fancyvrb}
\newcommand{\VerbBar}{|}
\newcommand{\VERB}{\Verb[commandchars=\\\{\}]}
\DefineVerbatimEnvironment{Highlighting}{Verbatim}{commandchars=\\\{\}}
% Add ',fontsize=\small' for more characters per line
\usepackage{framed}
\definecolor{shadecolor}{RGB}{248,248,248}
\newenvironment{Shaded}{\begin{snugshade}}{\end{snugshade}}
\newcommand{\AlertTok}[1]{\textcolor[rgb]{0.94,0.16,0.16}{#1}}
\newcommand{\AnnotationTok}[1]{\textcolor[rgb]{0.56,0.35,0.01}{\textbf{\textit{#1}}}}
\newcommand{\AttributeTok}[1]{\textcolor[rgb]{0.77,0.63,0.00}{#1}}
\newcommand{\BaseNTok}[1]{\textcolor[rgb]{0.00,0.00,0.81}{#1}}
\newcommand{\BuiltInTok}[1]{#1}
\newcommand{\CharTok}[1]{\textcolor[rgb]{0.31,0.60,0.02}{#1}}
\newcommand{\CommentTok}[1]{\textcolor[rgb]{0.56,0.35,0.01}{\textit{#1}}}
\newcommand{\CommentVarTok}[1]{\textcolor[rgb]{0.56,0.35,0.01}{\textbf{\textit{#1}}}}
\newcommand{\ConstantTok}[1]{\textcolor[rgb]{0.00,0.00,0.00}{#1}}
\newcommand{\ControlFlowTok}[1]{\textcolor[rgb]{0.13,0.29,0.53}{\textbf{#1}}}
\newcommand{\DataTypeTok}[1]{\textcolor[rgb]{0.13,0.29,0.53}{#1}}
\newcommand{\DecValTok}[1]{\textcolor[rgb]{0.00,0.00,0.81}{#1}}
\newcommand{\DocumentationTok}[1]{\textcolor[rgb]{0.56,0.35,0.01}{\textbf{\textit{#1}}}}
\newcommand{\ErrorTok}[1]{\textcolor[rgb]{0.64,0.00,0.00}{\textbf{#1}}}
\newcommand{\ExtensionTok}[1]{#1}
\newcommand{\FloatTok}[1]{\textcolor[rgb]{0.00,0.00,0.81}{#1}}
\newcommand{\FunctionTok}[1]{\textcolor[rgb]{0.00,0.00,0.00}{#1}}
\newcommand{\ImportTok}[1]{#1}
\newcommand{\InformationTok}[1]{\textcolor[rgb]{0.56,0.35,0.01}{\textbf{\textit{#1}}}}
\newcommand{\KeywordTok}[1]{\textcolor[rgb]{0.13,0.29,0.53}{\textbf{#1}}}
\newcommand{\NormalTok}[1]{#1}
\newcommand{\OperatorTok}[1]{\textcolor[rgb]{0.81,0.36,0.00}{\textbf{#1}}}
\newcommand{\OtherTok}[1]{\textcolor[rgb]{0.56,0.35,0.01}{#1}}
\newcommand{\PreprocessorTok}[1]{\textcolor[rgb]{0.56,0.35,0.01}{\textit{#1}}}
\newcommand{\RegionMarkerTok}[1]{#1}
\newcommand{\SpecialCharTok}[1]{\textcolor[rgb]{0.00,0.00,0.00}{#1}}
\newcommand{\SpecialStringTok}[1]{\textcolor[rgb]{0.31,0.60,0.02}{#1}}
\newcommand{\StringTok}[1]{\textcolor[rgb]{0.31,0.60,0.02}{#1}}
\newcommand{\VariableTok}[1]{\textcolor[rgb]{0.00,0.00,0.00}{#1}}
\newcommand{\VerbatimStringTok}[1]{\textcolor[rgb]{0.31,0.60,0.02}{#1}}
\newcommand{\WarningTok}[1]{\textcolor[rgb]{0.56,0.35,0.01}{\textbf{\textit{#1}}}}
\usepackage{graphicx,grffile}
\makeatletter
\def\maxwidth{\ifdim\Gin@nat@width>\linewidth\linewidth\else\Gin@nat@width\fi}
\def\maxheight{\ifdim\Gin@nat@height>\textheight\textheight\else\Gin@nat@height\fi}
\makeatother
% Scale images if necessary, so that they will not overflow the page
% margins by default, and it is still possible to overwrite the defaults
% using explicit options in \includegraphics[width, height, ...]{}
\setkeys{Gin}{width=\maxwidth,height=\maxheight,keepaspectratio}
\setlength{\emergencystretch}{3em}  % prevent overfull lines
\providecommand{\tightlist}{%
  \setlength{\itemsep}{0pt}\setlength{\parskip}{0pt}}
\setcounter{secnumdepth}{-2}
% Redefines (sub)paragraphs to behave more like sections
\ifx\paragraph\undefined\else
  \let\oldparagraph\paragraph
  \renewcommand{\paragraph}[1]{\oldparagraph{#1}\mbox{}}
\fi
\ifx\subparagraph\undefined\else
  \let\oldsubparagraph\subparagraph
  \renewcommand{\subparagraph}[1]{\oldsubparagraph{#1}\mbox{}}
\fi

% set default figure placement to htbp
\makeatletter
\def\fps@figure{htbp}
\makeatother


\title{Studying Various Effects of Severe Weather Events}
\author{Aman Bhagat}
\date{3/26/2020}

\begin{document}
\maketitle

\hypertarget{synopsis}{%
\subsection{Synopsis}\label{synopsis}}

In this report we aim to describe the effects of storms and other severe
weather events which can cause both public health and economic problems
for commumnities and municipaltie. Many severe events can results in
fatalities, injuries and property damage, and preventing such outcomes
to the extent possible is a key concern.

This project involves exploring the U.S. National Oceanic and
Atmoshpheric Administration's (NOAA) storm database. This database
tracks characteristics of major storms and weather events in the United
States, including when and where they occur, as well as estimates of any
fatalities, injuries and property damage.

\hypertarget{loading-and-processing-the-raw-data}{%
\subsection{Loading and Processing the Raw
Data}\label{loading-and-processing-the-raw-data}}

From
\href{https://d396qusza40orc.cloudfront.net/repdata\%2Fdata\%2FStormData.csv.bz2}{Storm
Data} we obtain the the storm data which is an official publication of
the National Oceanic and Atmoshpheric Administration. The data recorded
in the file is from 1950 to 2011. We first read the data from the raw
csv file which is compressed via bz2 algorithm to reduce its size.

\begin{Shaded}
\begin{Highlighting}[]
\NormalTok{weather <-}\StringTok{ }\KeywordTok{read.csv}\NormalTok{(}\StringTok{"repdata_data_StormData.csv.bz2"}\NormalTok{)}
\end{Highlighting}
\end{Shaded}

After reading the data we check first few rows(there are 902297 rows) of
the dataset

\begin{Shaded}
\begin{Highlighting}[]
\KeywordTok{dim}\NormalTok{(weather)}
\end{Highlighting}
\end{Shaded}

\begin{verbatim}
## [1] 902297     37
\end{verbatim}

\begin{Shaded}
\begin{Highlighting}[]
\KeywordTok{head}\NormalTok{(weather[,}\DecValTok{1}\OperatorTok{:}\DecValTok{8}\NormalTok{])}
\end{Highlighting}
\end{Shaded}

\begin{verbatim}
##   STATE__           BGN_DATE BGN_TIME TIME_ZONE COUNTY COUNTYNAME STATE  EVTYPE
## 1       1  4/18/1950 0:00:00     0130       CST     97     MOBILE    AL TORNADO
## 2       1  4/18/1950 0:00:00     0145       CST      3    BALDWIN    AL TORNADO
## 3       1  2/20/1951 0:00:00     1600       CST     57    FAYETTE    AL TORNADO
## 4       1   6/8/1951 0:00:00     0900       CST     89    MADISON    AL TORNADO
## 5       1 11/15/1951 0:00:00     1500       CST     43    CULLMAN    AL TORNADO
## 6       1 11/15/1951 0:00:00     2000       CST     77 LAUDERDALE    AL TORNADO
\end{verbatim}

Here we can see that the Begin Dates are not in the correct DATE format
to process the data. So we converted the dates into the the correct date
format for the better analysis of the data.

\begin{Shaded}
\begin{Highlighting}[]
\NormalTok{weather}\OperatorTok{$}\NormalTok{BGN_DATE <-}\StringTok{ }\KeywordTok{as.Date}\NormalTok{(weather}\OperatorTok{$}\NormalTok{BGN_DATE, }\DataTypeTok{format =} \StringTok{"%m/%d/%Y"}\NormalTok{)}
\KeywordTok{str}\NormalTok{(weather}\OperatorTok{$}\NormalTok{BGN_DATE)}
\end{Highlighting}
\end{Shaded}

\begin{verbatim}
##  Date[1:902297], format: "1950-04-18" "1950-04-18" "1951-02-20" "1951-06-08" "1951-11-15" ...
\end{verbatim}

\hypertarget{questions}{%
\subsection{Questions}\label{questions}}

\begin{enumerate}
\def\labelenumi{\arabic{enumi}.}
\item
  Across the United States, which types of events (as indicated in the
  EVTYPE variable) are most harmful with respect to population health?
\item
  Across the United States, which types of events have the greatest
  economic consequences?
\end{enumerate}

\hypertarget{results}{%
\subsection{Results}\label{results}}

\begin{quote}
Question 1: Across the United States, which types of events (as
indicated in the EVTYPE variable) are most harmful with respect to
population health?
\end{quote}

In order to address the first question we first grouped our original
data by the Event type and the number of Injuries and Fatalities ocured
due to that specific Event. We found that most of the Injuries are under
1000 while there are 14 major Weather Events which caused more than 1000
casualties. But by looking at the total number of Injuries, we found
that Tornado has caused more casualties(i.e.~over 91000). So we ran some
explonatory Analysis to confirm that this is not an outlier and by that
we observed that in the early years the events were not well recorded
while Tornado was recorded correctly throughout the years and by this
conclusion we were not able to conclude that the total number of
Injuries caused by Tornado was an outlier.

\textbf{Data processing for the required Plot}

\begin{Shaded}
\begin{Highlighting}[]
\NormalTok{harmful.evtype <-}\StringTok{ }\NormalTok{weather }\OperatorTok\StringTok{ }\KeywordTok{group_by}\NormalTok{(EVTYPE) }\OperatorTok\StringTok{ }\KeywordTok{summarize}\NormalTok{(}\DataTypeTok{Injuries.Total =} \KeywordTok{sum}\NormalTok{(INJURIES))}
\NormalTok{Fatalities <-}\StringTok{ }\NormalTok{weather }\OperatorTok\StringTok{ }\KeywordTok{group_by}\NormalTok{(EVTYPE) }\OperatorTok\StringTok{ }\KeywordTok{summarize}\NormalTok{(}\DataTypeTok{Fatalities.Total =} \KeywordTok{sum}\NormalTok{(FATALITIES))}
\NormalTok{Fatalities <-}\StringTok{ }\NormalTok{Fatalities[}\KeywordTok{order}\NormalTok{(}\OperatorTok{-}\NormalTok{Fatalities}\OperatorTok{$}\NormalTok{Fatalities.Total),]}
\NormalTok{Fatalities <-}\StringTok{ }\NormalTok{Fatalities[}\DecValTok{1}\OperatorTok{:}\DecValTok{14}\NormalTok{,]}

\NormalTok{df3 <-}\StringTok{ }\NormalTok{harmful.evtype }\OperatorTok\StringTok{ }\KeywordTok{filter}\NormalTok{(Injuries.Total }\OperatorTok{>}\StringTok{ }\DecValTok{1000}\NormalTok{) }\CommentTok{#Only the events which caused above 1000 Injuries}
\end{Highlighting}
\end{Shaded}

\textbf{Result}

\begin{Shaded}
\begin{Highlighting}[]
\NormalTok{plot1 <-}\StringTok{ }\KeywordTok{ggplot}\NormalTok{(}\DataTypeTok{data =}\NormalTok{ df3 , }\KeywordTok{aes}\NormalTok{(}\DataTypeTok{y =}\NormalTok{ Injuries.Total , }\DataTypeTok{x =}\NormalTok{ EVTYPE, }\DataTypeTok{fill =}\NormalTok{ EVTYPE))}\OperatorTok{+}
\StringTok{    }\KeywordTok{geom_bar}\NormalTok{(}\DataTypeTok{stat =} \StringTok{"identity"}\NormalTok{)}\OperatorTok{+}
\StringTok{    }\KeywordTok{guides}\NormalTok{(}\DataTypeTok{fill =} \OtherTok{FALSE}\NormalTok{)}\OperatorTok{+}
\StringTok{    }\KeywordTok{ggtitle}\NormalTok{(}\StringTok{"Injuries caused by Top 14 Harmful Events"}\NormalTok{)}\OperatorTok{+}
\StringTok{    }\KeywordTok{xlab}\NormalTok{(}\StringTok{"Event Type"}\NormalTok{)}\OperatorTok{+}
\StringTok{    }\KeywordTok{ylab}\NormalTok{(}\StringTok{"Total number of Injuries"}\NormalTok{)}\OperatorTok{+}
\StringTok{    }\KeywordTok{theme}\NormalTok{(}\DataTypeTok{text =} \KeywordTok{element_text}\NormalTok{(}\DataTypeTok{size=}\DecValTok{8}\NormalTok{),}\DataTypeTok{axis.text.x=}\KeywordTok{element_text}\NormalTok{(}\DataTypeTok{angle =} \DecValTok{30}\NormalTok{,}\DataTypeTok{hjust =} \DecValTok{1}\NormalTok{))}\OperatorTok{+}
\StringTok{    }\KeywordTok{scale_y_continuous}\NormalTok{(}\DataTypeTok{labels =}\NormalTok{ comma)}

\NormalTok{plot2 <-}\StringTok{ }\KeywordTok{ggplot}\NormalTok{(}\DataTypeTok{data =}\NormalTok{ Fatalities , }\KeywordTok{aes}\NormalTok{(}\DataTypeTok{y =}\NormalTok{ Fatalities.Total , }\DataTypeTok{x =}\NormalTok{ EVTYPE, }\DataTypeTok{fill =}\NormalTok{ EVTYPE))}\OperatorTok{+}
\StringTok{    }\KeywordTok{geom_bar}\NormalTok{(}\DataTypeTok{stat =} \StringTok{"identity"}\NormalTok{)}\OperatorTok{+}
\StringTok{    }\KeywordTok{guides}\NormalTok{(}\DataTypeTok{fill =} \OtherTok{FALSE}\NormalTok{)}\OperatorTok{+}
\StringTok{    }\KeywordTok{ggtitle}\NormalTok{(}\StringTok{"Fatalities caused by Top 14 Harmful Events"}\NormalTok{)}\OperatorTok{+}
\StringTok{    }\KeywordTok{xlab}\NormalTok{(}\StringTok{"Event Type"}\NormalTok{)}\OperatorTok{+}
\StringTok{    }\KeywordTok{ylab}\NormalTok{(}\StringTok{"Total number of Fatalities"}\NormalTok{)}\OperatorTok{+}
\StringTok{    }\KeywordTok{theme}\NormalTok{(}\DataTypeTok{text =} \KeywordTok{element_text}\NormalTok{(}\DataTypeTok{size=}\DecValTok{8}\NormalTok{),}\DataTypeTok{axis.text.x=}\KeywordTok{element_text}\NormalTok{(}\DataTypeTok{angle =} \DecValTok{30}\NormalTok{, }\DataTypeTok{hjust =} \DecValTok{1}\NormalTok{))}\OperatorTok{+}
\StringTok{    }\KeywordTok{scale_y_continuous}\NormalTok{(}\DataTypeTok{labels =}\NormalTok{ comma)}

\KeywordTok{grid.arrange}\NormalTok{(plot1 , plot2 , }\DataTypeTok{nrow =} \DecValTok{2}\NormalTok{)}
\end{Highlighting}
\end{Shaded}

\includegraphics{weathereffects_files/figure-latex/plot1-1.pdf}

\emph{The above plot show that there are 14 major weather events which
cause more Injuries and fatalities and TORNADO has caused more Injuries
and fatalities till 2011.}

\begin{quote}
Question 2: Across the United States, which types of events have the
greatest economic consequences?
\end{quote}

As to address the above question we first process the data to get the
required data which can be used for plotting and analysis purpose. Here
to calaculate total damage caused by the weather events we combine the
property damage and crop damage. There are some rows which doesn't have
proper record of the damage, so we only selected the rows which have
proper damage reported and then analyzed to get the top events which has
caused most damage and have the greatest economic consequences.

\textbf{Processing Data}

\begin{Shaded}
\begin{Highlighting}[]
\NormalTok{EconomicEvent <-}\StringTok{ }\NormalTok{weather }\OperatorTok\StringTok{ }\KeywordTok{select}\NormalTok{(EVTYPE,PROPDMG,PROPDMGEXP,CROPDMG,CROPDMGEXP)}

\CommentTok{#Filtering all the property damage and crop damage}
\NormalTok{damage <-}\StringTok{ }\NormalTok{EconomicEvent }\OperatorTok\StringTok{ }\KeywordTok{filter}\NormalTok{(PROPDMGEXP }\OperatorTok{==}\StringTok{ "K"} \OperatorTok{|}\StringTok{ }\NormalTok{PROPDMGEXP }\OperatorTok{==}\StringTok{ "k"} \OperatorTok{|}\StringTok{ }\NormalTok{PROPDMGEXP }\OperatorTok{==}\StringTok{ "m"} \OperatorTok{|}\StringTok{ }\NormalTok{PROPDMGEXP }\OperatorTok{==}\StringTok{ "M"} \OperatorTok{|}\StringTok{ }\NormalTok{PROPDMGEXP }\OperatorTok{==}\StringTok{ "b"} \OperatorTok{|}\StringTok{ }\NormalTok{PROPDMGEXP }\OperatorTok{==}\StringTok{ "B"}\NormalTok{ )}

\NormalTok{damage <-}\StringTok{ }\NormalTok{damage }\OperatorTok\StringTok{ }\KeywordTok{filter}\NormalTok{(CROPDMGEXP }\OperatorTok{==}\StringTok{ "K"} \OperatorTok{|}\StringTok{ }\NormalTok{CROPDMGEXP }\OperatorTok{==}\StringTok{ "k"} \OperatorTok{|}\StringTok{ }\NormalTok{CROPDMGEXP }\OperatorTok{==}\StringTok{ "m"} \OperatorTok{|}\StringTok{ }\NormalTok{CROPDMGEXP }\OperatorTok{==}\StringTok{ "M"} \OperatorTok{|}\StringTok{ }\NormalTok{CROPDMGEXP }\OperatorTok{==}\StringTok{ "b"} \OperatorTok{|}\StringTok{ }\NormalTok{CROPDMGEXP }\OperatorTok{==}\StringTok{ "B"}\NormalTok{ )}

\CommentTok{#Converting to mumeric value of the expression}
\NormalTok{damage}\OperatorTok{$}\NormalTok{PROPDMGEXP <-}\StringTok{ }\KeywordTok{gsub}\NormalTok{(}\StringTok{"k"}\NormalTok{ , }\DecValTok{1000}\NormalTok{, damage}\OperatorTok{$}\NormalTok{PROPDMGEXP,}\DataTypeTok{ignore.case =} \OtherTok{TRUE}\NormalTok{)}
\NormalTok{damage}\OperatorTok{$}\NormalTok{PROPDMGEXP <-}\StringTok{ }\KeywordTok{gsub}\NormalTok{(}\StringTok{"m"}\NormalTok{ , }\FloatTok{1e+06}\NormalTok{, damage}\OperatorTok{$}\NormalTok{PROPDMGEXP,}\DataTypeTok{ignore.case =} \OtherTok{TRUE}\NormalTok{)}
\NormalTok{damage}\OperatorTok{$}\NormalTok{PROPDMGEXP <-}\StringTok{ }\KeywordTok{gsub}\NormalTok{(}\StringTok{"b"}\NormalTok{ , }\FloatTok{1e+09}\NormalTok{, damage}\OperatorTok{$}\NormalTok{PROPDMGEXP, }\DataTypeTok{ignore.case =} \OtherTok{TRUE}\NormalTok{)}
\NormalTok{damage}\OperatorTok{$}\NormalTok{CROPDMGEXP <-}\StringTok{ }\KeywordTok{gsub}\NormalTok{(}\StringTok{"k"}\NormalTok{ , }\DecValTok{1000}\NormalTok{, damage}\OperatorTok{$}\NormalTok{CROPDMGEXP, }\DataTypeTok{ignore.case =} \OtherTok{TRUE}\NormalTok{)}
\NormalTok{damage}\OperatorTok{$}\NormalTok{CROPDMGEXP <-}\StringTok{ }\KeywordTok{gsub}\NormalTok{(}\StringTok{"m"}\NormalTok{ , }\FloatTok{1e+06}\NormalTok{, damage}\OperatorTok{$}\NormalTok{CROPDMGEXP, }\DataTypeTok{ignore.case =} \OtherTok{TRUE}\NormalTok{)}
\NormalTok{damage}\OperatorTok{$}\NormalTok{CROPDMGEXP <-}\StringTok{ }\KeywordTok{gsub}\NormalTok{(}\StringTok{"b"}\NormalTok{ , }\FloatTok{1e+09}\NormalTok{, damage}\OperatorTok{$}\NormalTok{CROPDMGEXP, }\DataTypeTok{ignore.case =} \OtherTok{TRUE}\NormalTok{)}
\NormalTok{damage}\OperatorTok{$}\NormalTok{CROPDMGEXP <-}\StringTok{ }\KeywordTok{as.numeric}\NormalTok{(damage}\OperatorTok{$}\NormalTok{CROPDMGEXP)}
\NormalTok{damage}\OperatorTok{$}\NormalTok{PROPDMGEXP <-}\StringTok{ }\KeywordTok{as.numeric}\NormalTok{(damage}\OperatorTok{$}\NormalTok{PROPDMGEXP)}

\CommentTok{#Calculating Damage by Event}
\NormalTok{damage}\OperatorTok{$}\NormalTok{Total.Damage <-}\StringTok{ }\NormalTok{(damage}\OperatorTok{$}\NormalTok{PROPDMG }\OperatorTok{*}\StringTok{ }\NormalTok{damage}\OperatorTok{$}\NormalTok{PROPDMGEXP) }\OperatorTok{+}\StringTok{ }\NormalTok{(damage}\OperatorTok{$}\NormalTok{CROPDMG }\OperatorTok{*}\StringTok{ }\NormalTok{damage}\OperatorTok{$}\NormalTok{CROPDMGEXP)}

\NormalTok{damage.by.event <-}\StringTok{ }\NormalTok{damage }\OperatorTok\StringTok{ }\KeywordTok{group_by}\NormalTok{(EVTYPE) }\OperatorTok\StringTok{ }\KeywordTok{summarize}\NormalTok{(}\DataTypeTok{Damage.Total =} \KeywordTok{sum}\NormalTok{(Total.Damage))}
\NormalTok{damage.by.event <-}\StringTok{ }\NormalTok{damage.by.event[}\KeywordTok{order}\NormalTok{(}\OperatorTok{-}\NormalTok{damage.by.event}\OperatorTok{$}\NormalTok{Damage.Total),]}
\NormalTok{damage.by.event <-}\StringTok{ }\NormalTok{damage.by.event[}\DecValTok{1}\OperatorTok{:}\DecValTok{5}\NormalTok{,]}
\end{Highlighting}
\end{Shaded}

\textbf{Results}

\begin{Shaded}
\begin{Highlighting}[]
\KeywordTok{ggplot}\NormalTok{(}\DataTypeTok{data =}\NormalTok{ damage.by.event , }\KeywordTok{aes}\NormalTok{(}\DataTypeTok{y =}\NormalTok{ Damage.Total , }\DataTypeTok{x =}\NormalTok{ EVTYPE, }\DataTypeTok{fill =}\NormalTok{ EVTYPE))}\OperatorTok{+}
\StringTok{    }\KeywordTok{geom_bar}\NormalTok{(}\DataTypeTok{stat =} \StringTok{"identity"}\NormalTok{)}\OperatorTok{+}
\StringTok{    }\KeywordTok{guides}\NormalTok{(}\DataTypeTok{fill =} \OtherTok{FALSE}\NormalTok{)}\OperatorTok{+}
\StringTok{    }\KeywordTok{ggtitle}\NormalTok{(}\StringTok{"Greatest Economic Consequences of Top 5 Harmful Events"}\NormalTok{)}\OperatorTok{+}
\StringTok{    }\KeywordTok{xlab}\NormalTok{(}\StringTok{"Event Type"}\NormalTok{)}\OperatorTok{+}
\StringTok{    }\KeywordTok{ylab}\NormalTok{(}\StringTok{"Total Damages"}\NormalTok{)}\OperatorTok{+}
\StringTok{    }\KeywordTok{theme}\NormalTok{(}\DataTypeTok{text =} \KeywordTok{element_text}\NormalTok{(}\DataTypeTok{size=}\DecValTok{8}\NormalTok{),}\DataTypeTok{axis.text.x=}\KeywordTok{element_text}\NormalTok{(}\DataTypeTok{angle =} \DecValTok{30}\NormalTok{, }\DataTypeTok{hjust =} \DecValTok{1}\NormalTok{))}\OperatorTok{+}
\StringTok{    }\KeywordTok{scale_y_continuous}\NormalTok{(}\DataTypeTok{labels =}\NormalTok{ comma)}
\end{Highlighting}
\end{Shaded}

\includegraphics{weathereffects_files/figure-latex/plot3-1.pdf}

\emph{From the above bar graph we can infer that Flood has caused the
most economic consequences.}

\hypertarget{summary}{%
\section{Summary}\label{summary}}

\emph{Hence from the above analysis on the given data, we can conclude
that Flood has caused more damage in terms of property and crop damage
all together. And from the initial analysis we can infer that tornado
has caused more damage in terms of public health and fatalities.}

\end{document}
